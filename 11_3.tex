\documentclass[a4,12pt]{article}
%\documentclass[a4,11pt]{jsarticle}
\usepackage{amsmath,amsthm,amssymb}
\usepackage{mathtools}
\usepackage{ascmac}
\usepackage{empheq}
%\usepackage[dvips]{graphicx}
\usepackage[dvipdfmx]{graphicx}
\usepackage{wrapfig}

\newtheorem{thm}{定理}
\newtheorem{dfn}{定義}
\newtheorem{prf}{証明}
\newtheorem{hyp}{仮定}
\newtheorem{rem}{注意}
\newtheorem{lem}{補題}

\renewcommand{\baselinestretch}{1.3}
%\renewcommand{\arraystretch}{2}
\topmargin=-1cm
\textheight=22cm
\oddsidemargin=0cm
\textwidth=16cm

\title{境界における反射による誤差評価}
\author{九澤俊介}
\date{2021年11月5日}
\begin{document}
\maketitle
\section{問題}
$T>0,\ t\in[0,T],\;f \in C^{0} ([0,T] \times \mathbb{R}^d \times \mathbb{R}^d )$に対し以下の微分方程式$x(t),\; v(t):[0,T]\rightarrow\mathbb{R}^d$について考える.ただし$x_0,v_0$は与えられているものとする.
\begin{equation}
\begin{cases}
\dot{x}(t)=v(t)\\
\dot{v}(t)=f(t,x(t),v(t))\\
x(0)=x_0\\
v(0)=v_0
\end{cases}
\end{equation}
上記の問題に対して$y(t),g(t,y(t))$を以下のように定義する.\\ただし$y(t):[0,T]\rightarrow\mathbb{R}^d\times\mathbb{R}^d,\;g\in C^0([0,T]\times\mathbb{R}^d)$.
\begin{equation}
\begin{cases}
y(t)\coloneqq (x(t),v(t))^T\\
g(t,y(t))\coloneqq (v(t),f(t,y(t)))^T
\end{cases}
\end{equation}
\begin{equation}
\begin{cases}
\dot{y}(t)=g(t,y(t))\\
y(0)=y_0
\end{cases}
\end{equation}
とし$y(t)$についての問題とする.
\section{数値解法}
$N\in\mathbb{N},T>0$に対して
\begin{equation}
\begin{cases}
\tau\coloneqq \lfloor\frac{T}{N}\rfloor \\
t_k\coloneqq k\tau\ (k=0,1,\ldots ,N)\\
y^k\sim y(t_k) \ (k=0,1,\ldots ,N)
\end{cases}
\end{equation}
$G(t_k,y^k;\tau)$を$1$段階法のschemeとして数値解$y^k$を繰り返し求めていく.
\begin{equation}
\begin{cases}\displaystyle\frac{y^{k+1}-y^k}{\tau}=G(t_k,y^k;\tau)\\
y^0:ginen
\end{cases}
\end{equation}
ただし$G(t_k,y^k;\tau)$は以下のリプシッツ条件を仮定する.
\begin{hyp}
\label{lip}
\[{}^\exists\tau_0 > 0\; {}^\exists L>0\; s.t.\; |G(t,y;\tau)-G(t,z;\tau)|\leq L|y-z|\]
\[(0\leq t\leq T,\  y,z\in\mathbb{R}^{2d},\tau\in(0,\tau_0))\]
\end{hyp}
\subsection{1段階法}
$G(t_k,y^k;\tau)$が以下の場合\\
\\(i)オイラー法の場合
\[G(t_k,y^k;\tau)=g(t_k,y^k)\]
(ii)ルンゲクッタ法の場合
\[G(t_k,y^k;\tau)=\frac{1}{6}(\ell_1+2\ell_2+2\ell_3+\ell_4)\]
\[\begin{cases}
\ell_1=g(t_k,y^k)\\
\ell_2=g(t_k+\displaystyle\frac{\tau}{2},y^k+\frac{\tau}{2}\ell_1)\\
\ell_3=g(t_k+\displaystyle\frac{\tau}{2},y^k+\frac{\tau}{2}\ell_2)\\ 
\ell_4=g(t_k,y^k+\tau \ell_3)
\end{cases}\]
\subsection{打ち切り誤差}
$T>0,t\in[0,T]$において$y(t)$を十分滑らかな関数とし、その数値解を$y$とする.この時$G(t,y;\tau)$の打ち切り誤差を$\epsilon(t;\tau)$とする.
\begin{dfn}\[\epsilon(t;\tau)\coloneqq\frac{y(t+\tau)-y(t)}{\tau}-G(t,y;\tau)\]\end{dfn}
$0\leq t\leq T-\tau$において以下の不等式が成り立つとき$G(t,y;\tau)$を$p$次の精度の近似法という.
\begin{equation}
{}^\exists C_0>0,\ |\epsilon(t;\tau)|\leq C_0\tau^p\;
\end{equation}
ただし$C_0$は$y(t)$に依存する.
\subsection{誤差}
厳密解と数値解の誤差を$r^k\coloneqq y(t_k)-y^k\ (k=0,1,\ldots,N)$とする.
\begin{thm}仮定$\ref{lip}$と(6)を満たすとき,$k=0,1,\ldots ,N,t\in[0,T]$に対して
\[|r^{k}|\leq |r_0|+C_0T\tau^{p}\]
が成り立つ.
\end{thm}
\begin{prf}
\[\begin{split}
\frac{r^{k+1}-r^k}{\tau}&=\frac{y(t_{k+1})-y^{k+1}}{\tau}- \frac{y(t_k)-y^k}{\tau}\\
&=\frac{y(t_k+\tau)-y(t_k)}{\tau}- \frac{y^{k+1}-y^k}{\tau}\\
&=\frac{y(t_k+\tau)-y(t_k)}{\tau}-G(t_k,y^k;\tau)\\
r^{k+1}&=r^k+\tau\epsilon(t^k;\tau)\\
|r^{k+1}|&=|r^k|+\tau |\epsilon(t^k;\tau)|\\
&\leq |r^k|+C_0\tau^{p+1}\\
\therefore |r^{k+1}|&\leq |r^k|+C_0\tau^{p+1}
\end{split}\]
$|r_0|$を用いると
\[\begin{split}
|r^{k+1}|&\leq |r_0|+(k+1)C_0\tau^{p+1}
\end{split}\]
したがって$|r^{k}|\leq |r_0|+kC_0\tau^{p+1}$より
\[|r^{k}|\leq |r_0|+C_0T\tau^{p}\]
さらに$|r_0|=0$とすると
\[|r^{k}|\leq C_0T\tau^{p}\]
\end{prf}
が成立する.\qed
\section{反射}
閉じた領域内を粒子が移動すると仮定する.その時粒子が領域の境界と接触し、跳ね返ることを反射とする.
以下速度べクトル$\dot{x}(t)$の境界に対する垂直方向の大きさが反射の前後で変化しない完全反射を前提に考える.\\
$T>0,\;\Omega$を十分滑らかな境界を持つ領域とする.$\nu$を境界上の法線外向き単位ベクトルとする.\\

$x(t)\in C^{0}([0,T];\overline{\Omega})$に対して
\[\begin{cases}
I_{\Omega}\coloneqq \{t\in[0,T];\ x(t)\in\Omega\}\\
I_{\partial\Omega}\coloneqq \{t\in(0,T);\ x(t)\in\partial\Omega\}
\end{cases}\]
とする.\\
以下では
\begin{equation}\left\{\begin{aligned}
&x \in C^0([0,T];\overline{\Omega})\cap W^{1,\infty}(0,T;\mathbb{R}^d) \cap C^2(I_\Omega ;\mathbb{R}^d) \\
&\dot{x} \in \text{BV}\ (0,T;\mathbb{R}^d)
\end{aligned}\right.\end{equation}
と仮定する.このとき、
\[{}^{\exists} v^{\pm}(t)=\begin{cases}
\dot{x} \ (t\in{I_\Omega}) \\
\displaystyle\lim_{\epsilon\to +0}\dot{x}(t\pm\epsilon)
\end{cases}\]
が定義できる.
\begin{dfn}[完全反射]
\label{reflect}
$x(t)$が$t_{\ast}\in I_{\partial\Omega}$で\\
\begin{equation}v^+ (t_\ast)=v^{-}(t_\ast)-2(v^{-}(t_\ast)\cdot\nu (x(t_\ast)))\nu (x(t_\ast))\end{equation}
を満たすとき、$x(t)$は$t=t_{\ast}$において完全反射するという.
\end{dfn}
$f \in C^{0} ([0,T] \times \mathbb{R}^d \times \mathbb{R}^d ),x^0\in\Omega$とする.$x(t)$を$(\ast)$と
\begin{equation}
\begin{cases}
\ddot{x}(t)=f(t,x(t),\dot{x}(t))\ (t\in I_{\Omega})\\
x(0)=x^0\\
t_\ast\in I_{\Omega}\text{で完全反射条件}(\text{定義}\ref{reflect})\text{が成り立つ}
\end{cases}
\end{equation}
を満たすとする.このとき、$x(t)$ を近似する差分法を構成せよ.
$G$は$(6)$を満たすとする.
\begin{equation}
\begin{cases}
\displaystyle\frac{\hat{y}^{k+1}-y^k}{\tau}=G(t_k,y^k;\tau)\ (k=0,\ldots ,\lfloor\frac{T}{\tau}\rfloor-1)\\
y^0:given\\
y^{k+1}=
\begin{cases}
y^{k+1}\ &(\text{if}\ \hat{x}^{k+1}\in\overline{\Omega})\ (k=0,\ldots ,\lfloor\frac{T}{\tau}\rfloor-1)\\
R(y^k,\hat{y}^{k+1})\ &(\text{if}\ \hat{x}^{k+1}\notin\overline{\Omega})\ (k=0,\ldots ,\lfloor\frac{T}{\tau}\rfloor-1)
\end{cases}
\end{cases}
\end{equation}
の形の差分法を考える.
\begin{hyp}
\[\begin{cases}
I_{\Omega}= \{s_1\},s_1\in (0,T)\\
v^{-}(s_1)\cdot\nu(x(s_1))>0
\end{cases}\]
\end{hyp}
$y_0(t):[0,s_1+\alpha]\rightarrow\mathbb{R}^{2d}$は次の方程式の解とする.ただし$0\leq t<s_1$.
\begin{equation}\left\{\begin{aligned}
    &\dot{y}_0(t)=f(t,y_0(t))\\
    &y(s_1)=(x(s_1),v^{-}(s_1))
    \end{aligned}\right.
\end{equation}
$y_1(t):[s_1+\alpha,T]\rightarrow\mathbb{R}^{2d}$は次の方程式の解とする.ただし$s_1<t\leq T$.
\begin{equation}\left\{\begin{aligned}
    &\dot{y}_1(t)=f(t,y_1(t))\\
    &y(s_1)=(x(s_1),v^{+}(s_1))
    \end{aligned}\right.
\end{equation}
ここで、この$y_0,y_1$は次を満たすことに注意する.
\[y(t)=
\begin{cases}
y_0(t)\ (0\leq t<s_1)\\
y_1(t)\ (s_1<t\leq T)
\end{cases}
\]
\[\begin{cases}
\displaystyle\frac{\tilde{y}^{k+1}-\tilde{y}^k}{\tau}=G(t_k,\tilde{y}^k;\tau)\ (k=0,\ldots ,\lfloor\frac{s_1+\alpha}{\tau}\rfloor-1)\\
\tilde{y}^0:y^*\\
\end{cases}\]
は$y_0$に収束.\\
\begin{dfn}[符号付き距離関数]
境界を$\Gamma$とする.\[
ds(x,\Gamma)\coloneqq
\begin{cases}
+dist(x,\Gamma)\ (x\notin\bar{\Omega})\\
-dist(x,\Gamma)\ (x\in\bar{\Omega})
\end{cases}
\]\end{dfn}
\begin{equation}
    \begin{cases}
    \Gamma : C^n\text{級}\Rightarrow ds\in C^n(N^{\epsilon}(\Gamma)),\ N^\epsilon(\Gamma)\coloneqq \{x\in\mathbb{R}^d|dist(x,\Gamma)<\epsilon\}\\
    |ds(x)-ds(z)|\leq|x-z|
    \end{cases}
\end{equation}
\begin{equation}\begin{cases}
ds(\tilde{x},\Gamma)\in[-\epsilon,0)&\Rightarrow\ \tilde{x}^k\in\Omega\cap N^\epsilon(\Gamma)\\
ds(\tilde{x},\Gamma)\in(0,\epsilon]&\Rightarrow\ \tilde{x}^k\in N^\epsilon(\Gamma)\backslash \bar{\Omega}
\end{cases}\end{equation}

$ds(x^{-}_1(t))\in N^\epsilon(\Gamma)$となるような$t$に対し,$ds(x^{-}_1(t))$について$t=s_1$周りでテイラー展開すると,
\[ds(x^{-}_1(t))=ds(x^{-}_1(s_1))+(t-s_1)\dot{\varphi}(s_1)+\frac{1}{2}(t-s_1)^2\ddot{\varphi}(t_\ast)\]   
\begin{align*}\dot{\varphi}(s_1)&=\nabla ds(x^{-}_1(s_1))\cdot v^{-}_1(s_1)\\
&=\nu(x^{-}_1(s_1))\cdot v^{-}_1(s_1)
\end{align*}
$\beta_1\coloneqq \dot{\varphi}(s_1)$とし,$ds(x^{-}_1(s_1))=0$ということから
\[
    ds(x^{-}_1(t))=(t-s_1)\{\beta_1+\frac{1}{2}(t-s_1)\ddot{\varphi}(t_\ast)\}
\]
このとき,ある$\delta>0$が存在し
\[|t-s_1|\leq\delta\Rightarrow \beta_1+\frac{1}{2}(t-s_1)\ddot{\varphi}(t_\ast)\ge\frac{1}{2}\beta_1\]
が成り立つ.\\
よって$s_1-\delta\leq t\leq s_1$のとき
\begin{equation}\label{beta} ds(x^{-}_1(t))\leq\frac{1}{2}\beta_1(t-s_1)\ (<0)\end{equation}
$ds(\tilde{x}^k)$について$s_1-\delta\leq t\leq s_1$のとき
\[\begin{split}
ds(\tilde{x}^k) &\leq ds(x^{-}_1 (t_k))+| ds(\tilde{x}^k)-ds(x^{-}_1 (t_k)) | \\
&\leq \frac{1}{2}\beta _1 (t_k-s_1)+|y^\ast -\tilde{y} ^\ast |+C_1[y^{-}_1]\tau^p\ ((\ref{beta})\text{式より})
\end{split}\]
このとき
\begin{equation}\label{time_eq}
    |y^\ast -\tilde{y} ^\ast |+C_1[y^{-}_1]\tau^p\leq \displaystyle\frac{1}{4}\beta_1|(t_k-s_1)|
\end{equation}
が成り立つと仮定すると,
\[\begin{split}
ds(\tilde{x}^k) &\leq \frac{1}{2}\beta _1 (t_k-s_1)+|y^\ast -\tilde{y} ^\ast |+C_1[y^{-}_1]\tau^p\\
&< 0
\end{split}\]
したがって$(\ref{time_eq})$が成立するとき,すなわち$t_k\leq s_1-\displaystyle\frac{4}{\beta_1}(|y^\ast -\tilde{y} ^\ast |+C_1[y^{-}_1]\tau^p)$ならば,$ds(\tilde{x}^k)<0$より,
\[\tilde{x}^k \in \Omega\]
同様に,$s_1<t\leq s_1+\delta$のとき$t_k\geq s_1+\displaystyle\frac{4}{\beta_1}(|y^\ast -\tilde{y} ^\ast |+C_1[y^{-}_1]\tau^p)$ならば
$ds(\tilde{x}^k)>0$より
\[\tilde{x}^k \notin \Omega\]
\begin{hyp}
\[\begin{cases}
\delta^{\prime}=\displaystyle\frac{4}{\beta_1}(|y^\ast -\tilde{y} ^\ast |+C_1[y^{-}_1]\tau^p)<\delta\\
\tau < \delta
\end{cases}\]
\end{hyp}
\begin{lem}
${}^\exists k\in\mathbb{N}\ s.t.\ t_k=k\tau\in[s_1-\delta^{\prime},s_1+\delta^{\prime}]$
\[s.t.\ \begin{cases}\tilde{x}_\ell\in\Omega\ (\ell=0,1,\ldots,k)\\
\tilde{x}_{k+1}\in\bar{\Omega}
\end{cases}\]
特にこのとき$|s_1-t_k|\leq \delta^{\prime}$
\end{lem}
\end{document}