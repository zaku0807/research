\documentclass[a4,12pt]{article}
%\documentclass[a4,11pt]{jsarticle}
\usepackage{amsmath,amsthm,amssymb}
\usepackage{mathtools}
\usepackage{ascmac}
\usepackage{empheq}
%\usepackage[dvips]{graphicx}
\usepackage[dvipdfmx]{graphicx}
\usepackage{wrapfig}
\usepackage{otf}
\usepackage{enumerate}

\newtheorem{thm}{定理}
\newtheorem{dfn}{定義}
\newtheorem{prf}{証明}
\newtheorem{hyp}{仮定}
\newtheorem{rem}{注意}
\newtheorem{lem}{補題}

\renewcommand{\baselinestretch}{1.3}
%\renewcommand{\arraystretch}{2}
\topmargin=-1cm
\textheight=22cm
\oddsidemargin=0cm
\textwidth=16cm

\title{境界における反射による誤差評価}
\author{九澤俊介}
\date{2021年11月5日}
\begin{document}
\maketitle
\section{問題}
$T>0,\ t\in[0,T],\;f \in C^{0} ([0,T] \times \mathbb{R}^d \times \mathbb{R}^d )$に対し以下の微分方程式$x(t),\; v(t):[0,T]\rightarrow\mathbb{R}^d$について考える.ただし$x_0,v_0$は与えられているものとする.
\begin{equation}
\begin{cases}
\dot{x}(t)=v(t)\\
\dot{v}(t)=f(t,x(t),v(t))\\
x(0)=x_0\\
v(0)=v_0
\end{cases}
\end{equation}
上記の問題に対して$y(t),g(t,y(t))$を以下のように定義する.\\ただし$y(t):[0,T]\rightarrow\mathbb{R}^d\times\mathbb{R}^d,\;g\in C^0([0,T]\times\mathbb{R}^d)$.
\begin{equation}
\begin{cases}
y(t)\coloneqq (x(t),v(t))^T\\
g(t,y(t))\coloneqq (v(t),f(t,y(t)))^T
\end{cases}
\end{equation}
\begin{equation}
\begin{cases}
\dot{y}(t)=g(t,y(t))\\
y(0)=y_0
\end{cases}
\end{equation}
とし$y(t)$についての問題とする.
\section{数値解法}
$N\in\mathbb{N},T>0$に対して
\begin{equation}
\begin{cases}
\tau\coloneqq \lfloor\frac{T}{N}\rfloor \\
t_k\coloneqq k\tau\ (k=0,1,\ldots ,N)\\
y^k\sim y(t_k) \ (k=0,1,\ldots ,N)
\end{cases}
\end{equation}
$G(t_k,y^k;\tau)$を$1$段階法のschemeとして数値解$y^k$を繰り返し求めていく.
\begin{equation}
\begin{cases}\displaystyle\frac{y^{k+1}-y^k}{\tau}=G(t_k,y^k;\tau)\\
y^0:y^*
\end{cases}
\end{equation}
ただし$G(t_k,y^k;\tau)$は以下のリプシッツ条件を仮定する.
\begin{hyp}
\label{lip}
\[{}^\exists\tau_0 > 0\; {}^\exists L>0\; s.t.\; |G(t,y;\tau)-G(t,z;\tau)|\leq L|y-z|\]
\[(0\leq t\leq T,\  y,z\in\mathbb{R}^{2d},\tau\in(0,\tau_0))\]
\end{hyp}
\subsection{1段階法}
$G(t_k,y^k;\tau)$が以下の場合\\
\\(i)オイラー法の場合
\[G(t_k,y^k;\tau)=g(t_k,y^k)\]
(ii)ルンゲクッタ法の場合
\[G(t_k,y^k;\tau)=\frac{1}{6}(\ell_1+2\ell_2+2\ell_3+\ell_4)\]
\[\begin{cases}
\ell_1=g(t_k,y^k)\\
\ell_2=g(t_k+\displaystyle\frac{\tau}{2},y^k+\frac{\tau}{2}\ell_1)\\
\ell_3=g(t_k+\displaystyle\frac{\tau}{2},y^k+\frac{\tau}{2}\ell_2)\\ 
\ell_4=g(t_k,y^k+\tau \ell_3)
\end{cases}\]
\subsection{打ち切り誤差}
$T>0,t\in[0,T]$において$y(t)$を十分滑らかな関数とし、その数値解を$y$とする.この時$G(t,y;\tau)$の打ち切り誤差を$\epsilon(t;\tau)$とする.
\begin{dfn}\[\epsilon(t;\tau)\coloneqq\frac{y(t+\tau)-y(t)}{\tau}-G(t,y;\tau)\]\end{dfn}
次を満たす$t$と$\tau$に依存しない定数$C_G=C_G[y]$が存在するとき,$G$を$p$次の精度という.任意の$\tau>0,0<t<T-\tau$に対して
\begin{equation}
|\epsilon(t;\tau)|\leq C_G\tau^p
\end{equation}
(例)
\subsection{誤差}
厳密解と数値解の誤差を$r^k\coloneqq y(t_k)-y^k\ (k=0,1,\ldots,N)$とする.
\begin{thm}\label{th1}仮定$\ref{lip}$と(6)を満たすとき,$k=0,1,\ldots ,N,t\in[0,T]$に対して
\begin{equation}|r^{k}|\leq |r_0|+C_0T\tau^{p}\end{equation}
が成り立つ.
\end{thm}
\begin{prf}
\[\begin{split}
\frac{r^{k+1}-r^k}{\tau}&=\frac{y(t_{k+1})-y^{k+1}}{\tau}- \frac{y(t_k)-y^k}{\tau}\\
&=\frac{y(t_k+\tau)-y(t_k)}{\tau}- \frac{y^{k+1}-y^k}{\tau}\\
&=\frac{y(t_k+\tau)-y(t_k)}{\tau}-G(t_k,y^k;\tau)\\
r^{k+1}&=r^k+\tau\epsilon(t^k;\tau)\\
|r^{k+1}|&=|r^k|+\tau |\epsilon(t^k;\tau)|\\
&\leq |r^k|+C_0\tau^{p+1}\\
\therefore |r^{k+1}|&\leq |r^k|+C_0\tau^{p+1}
\end{split}\]
$|r_0|$を用いると
\[\begin{split}
|r^{k+1}|&\leq |r_0|+(k+1)C_0\tau^{p+1}
\end{split}\]
したがって$|r^{k}|\leq |r_0|+kC_0\tau^{p+1}$より
\[|r^{k}|\leq |r_0|+C_0T\tau^{p}\]
さらに$|r_0|=0$とすると
\[|r^{k}|\leq C_0T\tau^{p}\]
\end{prf}
が成立する.\qed
\section{反射}
閉じた領域内を粒子が移動すると仮定する.その時粒子が領域の境界と接触し、跳ね返ることを反射とする.
以下速度べクトル$\dot{x}(t)$の境界に対する垂直方向の大きさが反射の前後で変化しない完全反射を前提に考える.\\
$T>0,\;\Omega$を十分滑らかな境界を持つ領域とする.$\nu$を境界上の法線外向き単位ベクトルとする.\\

$x(t)\in C^{0}([0,T];\overline{\Omega})$に対して
\[\begin{cases}
I_{\Omega}\coloneqq \{t\in[0,T];\ x(t)\in\Omega\}\\
I_{\partial\Omega}\coloneqq \{t\in(0,T);\ x(t)\in\partial\Omega\}
\end{cases}\]
とする.\\
以下では
\begin{equation}\left\{\begin{aligned}
&x \in C^0([0,T];\overline{\Omega})\cap W^{1,\infty}(0,T;\mathbb{R}^d) \cap C^2(I_\Omega ;\mathbb{R}^d) \\
&\dot{x} \in \text{BV}\ (0,T;\mathbb{R}^d)
\end{aligned}\right.\end{equation}
と仮定する.このとき、
\[{}^{\exists} v^{\pm}(t)=\begin{cases}
\dot{x} \ (t\in{I_\Omega}) \\
\displaystyle\lim_{\epsilon\to +0}\dot{x}(t\pm\epsilon)
\end{cases}\]
が定義できる.
\begin{dfn}[完全反射]
\label{reflect}
$x(t)$が$t_{\ast}\in I_{\partial\Omega}$で\\
\begin{equation}v^+ (t_\ast)=v^{-}(t_\ast)-2(v^{-}(t_\ast)\cdot\nu (x(t_\ast)))\nu (x(t_\ast))\end{equation}
を満たすとき、$x(t)$は$t=t_{\ast}$において完全反射するという.
\end{dfn}
$f \in C^{0} ([0,T] \times \mathbb{R}^d \times \mathbb{R}^d ),x^0\in\Omega$とする.$x(t)$を$(\ast)$と
\begin{equation}
\begin{cases}
\ddot{x}(t)=f(t,x(t),\dot{x}(t))\ (t\in I_{\Omega})\\
x(0)=x^0\\
t_\ast\in I_{\Omega}\text{で完全反射条件が成り立つ}
\end{cases}
\end{equation}
を満たすとする.このとき、$x(t)$ を近似する差分法を構成せよ.
$G$は$(6)$を満たすとする.
\begin{equation}
\begin{cases}
\displaystyle\frac{\hat{y}^{k+1}-y^k}{\tau}=G(t_k,y^k;\tau)\ (k=0,\ldots ,\lfloor\frac{T}{\tau}\rfloor-1)\\
y^0:y^*\\
y^{k+1}=
\begin{cases}
y^{k+1}\ &(\text{if}\ \hat{x}^{k+1}\in\overline{\Omega})\ (k=0,\ldots ,\lfloor\frac{T}{\tau}\rfloor-1)\\
R(y^k,\hat{y}^{k+1})\ &(\text{if}\ \hat{x}^{k+1}\notin\overline{\Omega})\ (k=0,\ldots ,\lfloor\frac{T}{\tau}\rfloor-1)
\end{cases}
\end{cases}
\end{equation}
の形の差分法を考える.
\begin{hyp}
\[\begin{cases}
I_{\Omega}= \{s_1\},s_1\in (0,T),\\
v^{-}(s_1)\cdot\nu(x(s_1))>0.
\end{cases}\]
\end{hyp}
\medskip
$\alpha>0$とし、境界での反射前後で解の延長を考える.$y^{-}_1(t):[0,s_1+\alpha]\rightarrow\mathbb{R}^{2d}$は次の方程式の解とし,
\begin{equation}\left\{\begin{aligned}
    &\dot{y}^{-}_1(t)=f(t,y^{-}_1(t))\ \ (0<t<s_1+\alpha)\\
    &y(s_1)=(x(s_1),v^{-}(s_1))
    \end{aligned}\right.
\end{equation}
$y^{+}_1(t):[s_1+\alpha,T]\rightarrow\mathbb{R}^{2d}$は次の方程式の解とする.
\begin{equation}\left\{\begin{aligned}
    &\dot{y}^{+}_1(t)=f(t,y^{+}_1(t))\ \ (s_1-\alpha<t<T)\\
    &y(s_1)=(x(s_1),v^{+}(s_1))
    \end{aligned}\right.
\end{equation}
ここで、この$y^-_1,y^+_1$は次を満たすことに注意する.
\[y(t)=
\begin{cases}
y^{-}_1(t)\ (0\leq t<s_1),\\
y^{+}_1(t)\ (s_1<t\leq T).
\end{cases}
\]
$y^+_1$を近似する差分法
\[\begin{cases}
\displaystyle\frac{\tilde{y}^{k+1}-\tilde{y}^k}{\tau}=G(t_k,\tilde{y}^k;\tau)\ (k=0,\ldots ,\lfloor\frac{s_1+\alpha}{\tau}\rfloor-1)\\
\tilde{y}^0\coloneqq y^*\\
\end{cases}\]
の解は定理\ref{th1}より$y^-_1$に収束する.
\begin{equation}
    |y^-_1(t_k)-\tilde{y}^k|\le |y^0-y^\ast|+C_G[y^-_1]T\tau^p
\end{equation}
    
\begin{dfn}[符号付き距離関数]
境界を$\Gamma$とする.\[
d_s(x)\coloneqq
\begin{cases}
+\operatorname{dist}(x,\Gamma)\ (x\notin\overline{\Omega})\\
-\operatorname{dist}(x,\Gamma)\ (x\in\overline{\Omega})
\end{cases}
\]\end{dfn}
\medskip
これはLipschitz連続である.つまり任意の$x,z\in\mathbb{R}^d$に対して,
\begin{equation}
    |d_s(x)-d_s(z)|\le|x-z|
\end{equation}
また$\Gamma$が$C^n$級のとき,ある$\epsilon>0$が存在して,
\begin{equation}
d_s\in C^n(N^\epsilon (\Gamma))
\end{equation}
を満たす.ただし,$N^\epsilon(\Gamma)\coloneqq\{x\in\mathbb{R}^d|\operatorname{dist}(x,\Gamma)<\epsilon\}$.
また次が成り立つ.
\begin{equation}\begin{cases}
    d_s(x)\in(-\epsilon,0)&\Rightarrow\ x\in\Omega\cap N^\epsilon(\Gamma)\\
    d_s(x)\in(0,\epsilon)&\Rightarrow\ x\in N^\epsilon(\Gamma)\backslash \overline{\Omega}
\end{cases}\end{equation}

\begin{lem}
\label{dom}
$\Gamma$が$C^3$級とする.さらに,ある$\xi>0$が存在し,$|y_0-y^\ast|<\xi\tau$を満たすとし,$p=1$とする.このとき,ある$\delta>0,\sigma>1$が存在して,$\tau<\frac{\delta}{1+\sigma}$のとき,
\begin{equation}
    \begin{cases}
        (i)\ t_k\in [s_1-\delta,s_1-\sigma\tau ] \Rightarrow \tilde{x}^k\in\Omega\\
        (ii)\ t_k\in [s_1+\sigma\tau, s_1+\delta ] \Rightarrow \tilde{x}^k\notin\Omega
    \end{cases}
\end{equation}
が成り立つ.\\
\end{lem}
\begin{proof}
    $x^-_1$は$t=s_1$で連続のなので,ある$\delta_1>0$が存在して,
    \begin{equation}
        |t-s_1|\leq\delta_1\Rightarrow x^{-}_1(t)\in N^\epsilon(\Gamma)
    \end{equation}
が成り立つ.この$\delta_1$を用いて次のように記号を準備する.
\begin{equation}\begin{cases}
    M\coloneqq \underset{s_1-\delta_1\le t\le s_1+\delta_1}{\max}|\ddot{\varphi}(t)|\\
    \beta_1\coloneqq \frac{d}{dt}d_s(x^-_1(t))|_{t=s_1}\\
    \delta\coloneqq \min(\delta_1,\frac{\beta_1}{M})\\
    r_0\coloneqq \xi + C_G[y^-_1]T\\
    \sigma\coloneqq \max\{\frac{4r_0}{\beta_1},1\}\\
\end{cases}\end{equation}
    $|t-s_1|<\delta_1$に対して,$x^-_1(t)\in N^\epsilon(\Gamma)$であり,(14)より$ds(x^-_1)$は$C^2$級.\\$d_s(x^{-}_1(t))\eqqcolon\varphi(t)$について$t=s_1$周りでテイラー展開すると,
    各$|t-s_1|<\delta_1$に対して,
    \[d_s(x^{-}_1(t))=d_s(x^{-}_1(s_1))+(t-s_1)\dot{\varphi}(s_1)+\frac{1}{2}(t-s_1)^2\ddot{\varphi}(t_\ast)\]   
ただし,$t_\ast\coloneqq (1-\theta)t+\theta s_1,\theta\in(0,1)$.\\
    $d_s(x^{-}_1(s_1))=0$ということから
    \[
        d_s(x^{-}_1(t))=(t-s_1)\{\beta_1+\frac{1}{2}(t-s_1)\ddot{\varphi}(t_\ast)\}
    \]
ここで,$|t-s_1|<\delta$のとき,
\begin{align*}
    \beta_1+\frac{1}{2}(t-s_1)\ddot{\varphi}(t_\ast)
    &\ge \beta_1-\frac{1}{2}|t-s_1||\ddot{\varphi(t_\ast)}|\\
    &\ge \beta_1-\frac{\delta}{2}M\\
    &\ge \beta_1-\frac{1}{2}\frac{\beta_1}{M}M=\frac{\beta_1}{2}
\end{align*}
以上より,(13)と(14)を用いて,(16)が導かれる.実際,
$\\(i)\ s_1-\delta<t_k\le s_1-\sigma\tau$の場合,
\[\begin{split}
    d_s(\tilde{x}^k)&=d_s(x^{-}_1(t_k))+d_s(\tilde{x}^k)-d_s(x^{-}_1(t_k))\\
                    &\le d_s(x^{-}_1(t_k))+|\tilde{x}^k-x^{-}_1(t_k)|\\
                    &\le (t_k-s_1)(\beta_1+\frac{1}{2}(t_k-s_1)\ddot{\varphi}(t_\ast))+|\tilde{y}^k-y^-_1(t_k)|\\
                    &\le\frac{\beta_1}{2}(t_k-s_1)+|y_0-y^\ast|+C_G[y^-_1]T\tau\\
                    &\le\frac{\beta_1}{2}(-\sigma\tau)+(\xi+C_G[y^-_1]T)\tau\\
                    &=-(\frac{\sigma\beta_1}{2}-r_0)\tau<0\ 
\end{split}\]
$(ii)\ s_1+\sigma\tau\le t_k <s_1+\delta$の場合
\[\begin{split}
    d_s(\tilde{x}^k)&=d_s(x^{+}_1(t_k))+d_s(\tilde{x}^k)-d_s(x^{+}_1(t_k))\\
                    &\ge d_s(x^{+}_1(t_k))-|x^{+}_1(t_k)-\tilde{x}^k|\\
                    &\ge(s_1-t_k)(\beta_1+\frac{1}{2}(t_k-s_1)\ddot{\varphi}(t_\ast))-|y^+_1(t_k)-\tilde{y}^k|\\
                    &\ge\frac{\beta_1}{2}(s_1-t_k)-|y_0-y^\ast|-C_G[y^+_1]T\tau\\
                    &\ge\frac{\beta_1}{2}(\sigma\tau)-(\xi+C_G[y^-_1]T)\tau\\
                    &=(\frac{\sigma\beta_1}{2}-r_0)\tau>0\ 
\end{split}\]
\end{proof}
$\\$ 補題1より
\[{}^\exists k\ s.t. 
\begin{cases}
    |t_k-s_1|\le(\sigma+1)\tau\\
    \{\tilde{x_\ell}\}_{\ell=0}^k \in\overline{\Omega},\ \tilde{x}^{k+1}\notin\overline{\Omega}
\end{cases}\]
$\\(i)\ 0\le\ell\le k,\ell\tau\le s_1$
\[\Rightarrow|x(\ell\tau)-x^\ell|=|x^{-}_1(\ell\tau)-\tilde{x}^\ell|\le r(\tau)\]
$(ii)\ 0\le\ell\le k,\ell\tau>s_1$
\[\begin{split}
    \Rightarrow|x(\ell\tau)-x^\ell|&=|x^{+}_1(\ell\tau)-\tilde{x}^\ell|\\
                        &\le|x^{+}_1(\ell\tau)-x^{+}_1(s_1)|+|x^{+}_1(s_1)-x^{+}_1(\ell\tau)|+|x^{+}_1(\ell\tau)-\tilde{x}^\ell|\\
                        &\le L\sigma\tau+L\sigma\tau+r(\tau)\\
                        &\le (2L\sigma+r_0)\tau
\end{split}\]
$\\(iii)\ \ell\le k+1$の場合
\[|y^\ell - y^{+}_1(t_\ell)|\le|y^{k+1}-y^+_1(t_{\ell+1})|+C_1[y^+_1]\tau^p\]
\newpage
\begin{enumerate}[(I).]
    \item$y^\ell-y^{-}(t_\ell)$の評価
    \[\Rightarrow|x(t_\ell)-x^\ell|=|x^{-}_1(t_\ell)-\tilde{x}^\ell|\le r(\tau)\]
    \item$s_1\le t_k$の場合
    \begin{enumerate}[(i).]
        \item$t_\ell\le k\tau$での$x^\ell,x^+(t_\ell)$の評価
        \begin{align*}
            \Rightarrow|x(\ell\tau)-x^\ell|&=|x^{+}_1(\ell\tau)-\tilde{x}^\ell|\\
                                &\le|x^{+}_1(\ell\tau)-x^{+}_1(s_1)|+|x^{+}_1(s_1)-x^{-}_1(\ell\tau)|+|x^{-}_1(\ell\tau)-\tilde{x}^\ell|\\
                                &\le L\sigma\tau+L\sigma\tau+r(\tau)\\
                                &\le (2L\sigma+r_0)\tau
            \end{align*}
        \item$t_\ell\ge (k+1)\tau$での$y^\ell,y^+(t_\ell)$の評価\\
            $(a).x^\ell,x^+(t_\ell)$の評価
            \begin{align*}
            |x^+_1(t_\ell)-x^{k+1}|&=|x^+_1(s_1)+\int^{t_\ell}_{s_1}v^+_1(t)dt-(x^\ast+\theta\tau(v^{k}-2(v^{k}\cdot\nu(x^\ast))\nu(x^\ast)))|\\
                          &\leq |x^+_1(s_1)-x^\ast|+|\theta\tau(v^{\ell-1}-2(v^{\ell-1}\cdot\nu(z))\nu(z))|+|\int^{t_\ell}_{s_1}v_1^+(t)dt|
            \end{align*}
            $(b).v^\ell,v^+(t_\ell)$の評価
            \begin{align*}
            |v(t_\ell)-v^\ell|&=|v_1^+(s_1)+\int^{(k+1)\tau}_{s_1}f(t,y(t))dt-(v^{\ell-1}-2(v^{\ell-1}\cdot\nu(z))\nu(z)))|\\
                              &\le |v_1^-(s_1)-v^{\ell-1}|+2|(v^{\ell-1}\cdot\nu(z))\nu(z))-(v_1^-(s_1)\cdot\nu(y(s_1)))\nu(y(s_1)))|\\
                              &\ \ \ \ \ \ \ +|\int^{(k+1)\tau}_{s_1}f(t,y_1^+(t))dt|
            \end{align*}
    \end{enumerate}
    \item$t_k<s_1$の場合
    \begin{enumerate}[(i).]
        \item $t_\ell>s_1$での$x^\ell,x^+(t_\ell)$の評価\[a=a\]
        \item $(k+1)\tau\le t_\ell<s_1$での$y^\ell,y^+(t_\ell)$の評価\[a=a\]
    \end{enumerate}
\end{enumerate}


\end{document}



